\section{Especificações do Equipamento utilizado}

A infraestrutura foi implementada em 3 computadores, sendo que todos os servidores tinham uma versão minimal do CentOS 7.0 de 64 bits. As máquinas possuem as caracteristicas que se apresentam de seguida.

\paragraphnl{Máquina 1}
\textbf{Sistema operativo nativo:} \textit{MAC OS YOSEMITE 10.10.2} \\
\textbf{Processador:} \textit{2,3 GHz Intel Core i5} \\
\textbf{Memoria:} \textit{8 GB 1333 MHz DDR3} \\
\textbf{Gráfica:} \textit{Intel HD Graphics 3000 512 MB} \\
\textbf{Tipo de armazenamento:} \textit{Samsung SSD 840 EVO 250GB} \\

\paragraphnl{Máquina 2}
\textbf{Sistema operativo nativo:} \textit{Ubuntu 14.04 LTS} \\
\textbf{Processador:} \textit{AMD A6-3420M APU a 1.4 GHz} \\
\textbf{Memoria:} \textit{6 GB} \\
\textbf{Gráfica:} \textit{Radeon HD Graphics} \\
\textbf{Tipo de armazenamento:} \textit{HDD 500GB} \\

\paragraphnl{Máquina 3}
\textbf{Sistema operativo nativo:} \textit{Windows 8.1} \\
\textbf{Processador:} \textit{i7-3630QM CPU @ 2.40 GHz} \\
\textbf{Memoria:} \textit{8GB} \\
\textbf{Gráfica:} \textit{Geforce GT 650M} \\
\textbf{Tipo de armazenamento:} \textit{HDD 500GB} \\

A máquina que irá estar ligada à infraestrutura a fazer benchmarking será a máquina 3. A Máquina 2 irá suportar todas os servidores WEB necessários. Já a máquina 1 está encarregue de disponibilizar os serviços como o \textit{Load Balancer}, armazenamento partilhado e o \textit{cluster}.
