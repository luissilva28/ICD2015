\section{Introdução}

Este projeto tem como objetivo operacionalizar um serviço escalável de geração de gráficos baseados em séries temporais.
Neste relatório irá ser abordado o projeto graphite que disponibiliza uma plataforma para a construção de graficos baseados em séries temporais.
Numa primeira fase serão explicadas as decisões tomadas relativamente à arquitetura da infraestrutura e a sua respetiva implementação.
De seguida, serão apresentados os resultados dos testes efetuados à infraestrutura.

\subsection{Motivações e objetivos}

A principal motivação na realização deste projeto, foca-se essencialmente na criação de uma infraestrutura, pois era algo que não tínhamos qualquer experiência à partida, e que no mundo real é extremamente importante. Por exemplo, no futuro caso seja necessário desenvolver uma aplicação de alta confiabilidade e escalabilidade, já temos os conhecimentos necessários, ou pelo menos, uma base sólida.
O objetivo deste trabalho é conceber uma infraestrutura de centro de dados para integrar uma aplicação, instalando e configurando todo o software/hardware, assim como realizar um conjunto de testes, para medir a sua qualidade.

\subsection{O Projeto Graphite}

O projeto Graphite é um sistema de gráficos em tempo real altamente escalável. O Graphite consiste em três componentes de software sendo elas \textbf{Carbon}, \textbf{Whisper} e \textbf{Web app}.
\textbf{Carbon} A sua principal tarefa é recolher dados estatisticos da rede e armazená-los em disco.
\textbf{Whisper} Esta componente é um formato de dados, ou seja, consiste num formato padrão para o armazenamento de dados de séries temporais.
\textbf{Web App} Consiste num Web app em Django, que fornece uma interface gráfica para o Graphite, ou seja, esta será a interface onde será possivel ver os gráficos pretendidos. Est Web App, permite tambem efetuar uma gestão dos utilizadores, configurando assim quais os usuários que podem criar e guardar os gráficos.
