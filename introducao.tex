\section{Resumo}
Nos dias que correm, existem muitas aplicações que necessitam de uma infraestrutura de centro de dados para suportar uma alta taxa de disponibilidade assim como um elevado nível de tolerância a falhas.

Com todas estas novas exigências do mercado, um administrador de centro de dados vê-se obrigado a efetuar toda a configuração quer de software, quer de hardware, bem como tomar as principais decisões sobre a infraestrutura com base no orçamento que tem disponivel para o efeito.

Neste documento apresentamos o projeto de Infraestrutura de Centro de Dados, onde de forma clara são ditos os procedimentos adotados na criação de uma infraestrutura de dados para uma aplicação, que neste caso passa pelo projeto Graphite. Além disso é apresentada uma série de testes efetuados que permitiram avaliar a qualidade da infraestrutura produzida.

\section{Introdução}
Este projeto tem como objetivo operacionalizar um serviço escalável de geração de gráficos baseados em séries temporais.
Neste relatório irá ser abordado o projeto graphite que disponibiliza uma plataforma para a construção de graficos baseados em séries temporais.
Numa primeira fase serão explicadas as decisões tomadas relativamente à arquitetura da infraestrutura e a sua respetiva implementação.
De seguida, serão apresentados os resultados dos testes efetuados à infraestrutura.

\subsection{Motivações e objetivos}

A principal motivação na realização deste projeto, foca-se essencialmente na criação de uma infraestrutura, pois era algo que não tínhamos qualquer experiência à partida, e que no mundo real é extremamente importante. Por exemplo, no futuro caso seja necessário desenvolver uma aplicação de alta confiabilidade e escalabilidade, já temos os conhecimentos
necessários, ou pelo menos, uma base sólida.
O objetivo deste trabalho é conceber uma infraestrutura de centro de dados para integrar uma aplicação, instalando e configurando todo o software/hardware, assim como realizar um conjunto de testes, para medir a sua qualidade.

\subsection{Estrutura do relatório}
Este relatório foi escrito de forma a ser facilmente compreensivel, para facilitar a leitura e replicação do estudo realizado.
Desta forma optou-se por dividir o relatório em capítulos diferentes, dividindo os mesmo em várias secções. Os capítulos principais deste relatório são os seguintes:

\textbf{Configuração do sistema} \textit{Neste primeiro ponto, serão descritos todos os passos efetuados para a instalação e configuração do ambiente de testes.}
\textbf{Infraestrutura computacional} \textit{Neste ponto serão apresentados os diferentes componentes da infraestrutura criada}
\textbf{Testes e análise de resultados} \textit{Neste ponto serão apresentados os vários testes efetuados bem como os resultados obtidos.}
\textbf{Conclusão} \textit{Neste ponto é feita uma avaliação crítica do projeto, apresentando as conclusões que achamos necessárias.}

\subsection{O Projeto Graphite}
O projeto Graphite é um sistema de gráficos em tempo real altamente escalável. O Graphite consiste em três componentes de software sendo elas \textbf{Carbon}, \textbf{Whisper} e \textbf{Web app}.
\textbf{Carbon} A sua principal tarefa é recolher dados estatisticos da rede e armazená-los em disco.
\textbf{Whisper} Esta componente é um formato de dados, ou seja, consiste num formato padrão para o armazenamento de dados de séries temporais.
\textbf{Web App} Consiste num Web app em Django, que fornece uma interface gráfica para o Graphite, ou seja, esta será a interface onde será possivel ver os gráficos pretendidos. Est Web App, permite tambem efetuar uma gestão dos utilizadores, configurando assim quais os usuários que podem criar e guardar os gráficos.

\subsubsection{Especificações do Equipamento utilizado}
A infraestrutura foi implementada em 3 computadores, sendo que todos os servidores tinham uma versão minimal do CentOS 7.0 de 64 bits. As máquinas possuem as caracteristicas que se apresentam de seguida.

\textbf{Máquina 1}
\textbf{Sistema operativo nativo:} \textit{}
\textbf{Processador:} \textit{}
\textbf{Memoria:} \textit{}
\textbf{Gráfica:} \textit{}
\textbf{Tipo de armazenamento:} \textit{}
\textbf{Capacidade de armazenamento da VM:} \textit{}
\textbf{Sistema operativo da VM:} \textit{}

\textbf{Máquina 2}
\textbf{Sistema operativo nativo:} \textit{}
\textbf{Processador:} \textit{}
\textbf{Memoria:} \textit{}
\textbf{Gráfica:} \textit{}
\textbf{Tipo de armazenamento:} \textit{}
\textbf{Capacidade de armazenamento da VM:} \textit{}
\textbf{Sistema operativo da VM:} \textit{}

\textbf{Máquina 3}
\textbf{Sistema operativo nativo:} \textit{}
\textbf{Processador:} \textit{}
\textbf{Memoria:} \textit{}
\textbf{Gráfica:} \textit{}
\textbf{Tipo de armazenamento:} \textit{}
\textbf{Capacidade de armazenamento da VM:} \textit{}
\textbf{Sistema operativo da VM:} \textit{}
