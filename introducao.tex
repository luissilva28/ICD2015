\section{Introdução}

Este projeto tem como objetivo operacionalizar um serviço escalável de geração de gráficos baseados em séries temporais.
Neste relatório irá ser abordado o projeto graphite que disponibiliza uma plataforma para a construção de graficos baseados em séries temporais.
Numa primeira fase serão explicadas as decisões tomadas relativamente à arquitetura da infraestrutura e a sua respetiva implementação.
De seguida, serão apresentados os resultados dos testes efetuados à infraestrutura.

\subsection{Motivações e objetivos}

A principal motivação na realização deste projeto, foca-se essencialmente na criação de uma infraestrutura, pois era algo que não tínhamos qualquer experiência à partida, e que no mundo real é extremamente importante. Por exemplo, no futuro caso seja necessário desenvolver uma aplicação de alta confiabilidade e escalabilidade, já temos os conhecimentos necessários, ou pelo menos, uma base sólida.
O objetivo deste trabalho é conceber uma infraestrutura de centro de dados para integrar uma aplicação, instalando e configurando todo o software/hardware, assim como realizar um conjunto de testes, para medir a sua qualidade.

\subsection{Estrutura do relatório}
Este relatório foi escrito de forma a ser facilmente compreensivel, para facilitar a leitura e replicação do estudo realizado.
Desta forma optou-se por dividir o relatório em capítulos diferentes, dividindo os mesmo em várias secções. Os capítulos principais deste relatório são os seguintes:

\begin{description}

\item[Configuração do sistema]
Neste primeiro ponto, serão descritos todos os passos efetuados para a instalação e configuração do ambiente de testes.

\item[Infraestrutura computacional]
Neste ponto serão apresentados os diferentes componentes da infraestrutura criada.

\item[Testes e análise de resultados]
Neste ponto serão apresentados os vários testes efetuados bem como os resultados obtidos.

\item[Conclusão]
Neste ponto é feita uma avaliação crítica do projeto, apresentando as conclusões que achamos necessárias.

\end{description}

\subsection{O Projeto Graphite}

O projeto Graphite é um sistema de gráficos em tempo real altamente escalável. O Graphite consiste em três componentes de software sendo elas \textbf{Carbon}, \textbf{Whisper} e \textbf{Web app}.
\textbf{Carbon} A sua principal tarefa é recolher dados estatisticos da rede e armazená-los em disco.
\textbf{Whisper} Esta componente é um formato de dados, ou seja, consiste num formato padrão para o armazenamento de dados de séries temporais.
\textbf{Web App} Consiste num Web app em Django, que fornece uma interface gráfica para o Graphite, ou seja, esta será a interface onde será possivel ver os gráficos pretendidos. Est Web App, permite tambem efetuar uma gestão dos utilizadores, configurando assim quais os usuários que podem criar e guardar os gráficos.

\subsubsection{Especificações do Equipamento utilizado}

A infraestrutura foi implementada em 3 computadores, sendo que todos os servidores tinham uma versão minimal do CentOS 7.0 de 64 bits. As máquinas possuem as caracteristicas que se apresentam de seguida.

\textbf{Máquina 1}
\textbf{Sistema operativo nativo:} \textit{}
\textbf{Processador:} \textit{}
\textbf{Memoria:} \textit{}
\textbf{Gráfica:} \textit{}
\textbf{Tipo de armazenamento:} \textit{}
\textbf{Capacidade de armazenamento da VM:} \textit{}
\textbf{Sistema operativo da VM:} \textit{}

\textbf{Máquina 2}
\textbf{Sistema operativo nativo:} \textit{}
\textbf{Processador:} \textit{}
\textbf{Memoria:} \textit{}
\textbf{Gráfica:} \textit{}
\textbf{Tipo de armazenamento:} \textit{}
\textbf{Capacidade de armazenamento da VM:} \textit{}
\textbf{Sistema operativo da VM:} \textit{}

\textbf{Máquina 3}
\textbf{Sistema operativo nativo: Windows 8.1} \textit{}
\textbf{Processador: i7-3630QM CPU @ 2.40 GHz} \textit{}
\textbf{Memoria: 8GB} \textit{}
\textbf{Gráfica: Geforce GT 650M} \textit{}
\textbf{Tipo de armazenamento:} \textit{}
\textbf{Capacidade de armazenamento da VM: 10Gb} \textit{}
\textbf{Sistema operativo da VM: CentOS 7} \textit{}
