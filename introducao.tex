\section{Introdução}
Este projeto tem como objetivo operacionalizar um serviço escalável de geração de gráficos baseados em séries temporais.
Neste projeto irá ser abordado o projeto graphite que disponibiliza uma plataforma para a construção de graficos baseados em séries temporais.

\subsection{Motivações e objetivos}

A principal motivação na realização deste projeto, foca-se essencialmente na criação de uma infraestrutura, pois era algo que não tínhamos qualquer experiência à partida, e que no mundo real é extremamente importante. Por exemplo, no futuro caso seja necessário desenvolver uma aplicação de alta confiabilidade e escalabilidade, já temos os conhecimentos
necessários, ou pelo menos, uma base sólida.
O objetivo deste trabalho é conceber uma infraestrutura de centro de dados para integrar uma aplicação, instalando e configurando todo o software/hardware, assim como realizar um conjunto de testes, para medir a sua qualidade.

\subsection{Estrutura do relatório}
Este relatório foi escrito de forma a ser facilmente compreensivel, para facilitar a leitura e replicação do estudo realizado.
Desta forma optou-se por dividir o relatório em capítulos diferentes, dividindo os mesmo em várias secções. Os capítulos principais deste relatório são os seguintes:

\textbf{Configuração do sistema} \textit{Neste primeiro ponto, serão descritos todos os passos efetuados para a instalação e configuração do ambiente de testes.}
\textbf{Infraestrutura computacional} \textit{Neste ponto serão apresentados os diferentes componentes da infraestrutura criada}
\textbf{Testes e análise de resultados} \textit{Neste ponto serão apresentados os vários testes efetuados bem como os resultados obtidos.}
\textbf{Conclusão} \textit{Neste ponto é feita uma avaliação crítica do projeto, apresentando as conclusões que achamos necessárias.}

\subsubsection{Especificações do Equipamento utilizado}
A infraestrutura estava implementada em 3 computadores, sendo que todos os servidores tinham uma versão minimal do CentOS 6.5 de 64 bits. As máquinas encontram-se com as caracteristicas que se apresentam de seguida.

\textbf{Máquina 1}
\textbf{Sistema operativo nativo:} \textit{}
\textbf{Processador:} \textit{}
\textbf{Memoria:} \textit{}
\textbf{Gráfica:} \textit{}
\textbf{Tipo de armazenamento:} \textit{}
\textbf{Capacidade de armazenamento da VM:} \textit{}
\textbf{Sistema operativo da VM:} \textit{}

\textbf{Máquina 2}
\textbf{Sistema operativo nativo:} \textit{}
\textbf{Processador:} \textit{}
\textbf{Memoria:} \textit{}
\textbf{Gráfica:} \textit{}
\textbf{Tipo de armazenamento:} \textit{}
\textbf{Capacidade de armazenamento da VM:} \textit{}
\textbf{Sistema operativo da VM:} \textit{}

\textbf{Máquina 3}
\textbf{Sistema operativo nativo:} \textit{}
\textbf{Processador:} \textit{}
\textbf{Memoria:} \textit{}
\textbf{Gráfica:} \textit{}
\textbf{Tipo de armazenamento:} \textit{}
\textbf{Capacidade de armazenamento da VM:} \textit{}
\textbf{Sistema operativo da VM:} \textit{}
