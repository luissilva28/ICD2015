\section{Configuração dos Servidores da Infraestrutura}

\subsection{Configuração Básica e Comum a Todos os Servidores}

Todos os servidores que fazem parte da infraestrutura adotada para esta projeto partilham uma base configurativa em comum.
Estas configurações devem ser vistas como configurações básicas de qualquer servidor, independentemente da função que este venha a desempenhar na arquitetura desenhada. \\

Por forma a diminuir o tempo de configuração de cada máquina da infraestrutura, optou-se pela configuração inicial de uma primeira máquina.
Todas os restantes servidores resultaram de um \textit{full clone} da primeira máquina que fora já configurada.
Uma vez que se está a trabalhar com máquinas virtuais, o processo de clonagem destas é levado a cabo pelo \textit{software} de virtualização, o \textit{VMWare}. \\

Cada um dos seguintes parágrafos diz respeito a um passo da configuração básica do servidor.

\paragraphnl{Definição do Nome do Servidor}

O nome do servidor é atributo mais relevante aquando da distinção deste entre os restantes.
E dado que se está a trabalhar com um número elevado de servidores, este é um aspeto ainda mais relevante.

A alteração do \textit{hostname} foi efetivada recorrendo à ferramenta \textit{hostnamectl}.

\begin{figure}[!hbt]
\begin{MyVerbatim}
$ hostnamectl set-hostname <NOME_DO_SERVIDOR>+
\end{MyVerbatim}
\end{figure}

\paragraphnl{Definição do IP Estático do Servidor}

Após a definição do nome do servidor, é igualmente importante associar um endereço IP da rede interna da infraestrutura que não possa variar ao longo do tempo, isto é, que seja estático.
Como é sabido, a infraestrutura funciona através de partilha de serviços entre das diversas camadas desta. \\

Assim, como estes serviços necessitam das referências aos endereços IPv4 dos servidores envolvidos, a constante variação destes endereços implicaria que os ficheiros de configuração dos serviços fossem, também eles, constatemente alterados. \\

O procedimento para a definição de um IP Estático é descrito em seguida.

\begin{figure}[!hbt]
\begin{MyVerbatim}
$ vi /etc/sysconfig/network-scripts/ifcfg-eno16777736
	BOOTPROTO="static"
	IPADDR="<IP_DO_SERVIDOR>"
\end{MyVerbatim}
\end{figure}

\newpage

\paragraphnl{Desativação do \textit{SE Linux} e da \textit{Firewall}}

Dado se tratar de um projeto de ãmbito académico, e uma vez que o \textit{SE Linux} limita bastante a operacionalidade de um servidor numa arquitetura deste tipo sem que seja necessária configuração adicional para este serviço, optou-se por desativar estes dois serviços. \\

No entato, se estivéssemos perante um ambiente real  de produção, o caminho a seguir seria o de configuração detalhada de ambos os serviços. \\

O procedimento para desativação do \textit{SE Linux} encontra-se descrito de seguida.

\begin{figure}[!hbt]
\begin{MyVerbatim}
$ vi /etc/selinux/config
	SELINUX=disabled
\end{MyVerbatim}
\end{figure}

Já quanto à desativação da \textit{firewall}, os comandos utilizados são apresentados a seguir.

\begin{figure}[!hbt]
\begin{MyVerbatim}
$ systemctl stop firewalld
$ systemctl mask firewalld
\end{MyVerbatim}
\end{figure}

\paragraphnl{Configuração do Acesso ao Servidor via \textit{SSH}}

A ferramenta \textit{SSH} permite aceder remotamente a um servidor.
Este acesso torna possível correr uma \textit{shell} neste. \\

O processo de autenticação do utilizador no servidor pode ser feito através do tradicional par \textit{username}/\textit{password} ou, de forma mais eficiente, recorrendo ao mecanismo de autenticação por chave pública.
Este último foi o selecionado como método de autenticação nos servidores da infraestrutura. \\

Assim, foi dado permissão de \textit{login} às chaves públicas dos utilizadores que podem ter acesso às máquinas.

\begin{figure}[!hbt]
\begin{MyVerbatim}
$ cat $HOME/.ssh/id_rsa.pub | \
ssh root@<IP_DO_SERVIDOR> "cat >> .ssh/authorized_keys"
\end{MyVerbatim}
\end{figure}

\newpage
\paragraphnl{Adição do Repositório \textit{Extra Packages for Enterprise Linux (EPEL)}}

A junção do repositório \textit{EPEL} ao conjunto de repositórios de pacotes de \textit{software} geridos pelas ferramenta \textit{yum} permite ter acesso a um conjunto muito mais vasto de programas e bibliotecas do que aquele que os repositórios por defeito oferecem.
Entre estes pacotes está, a título de exemplo, o software necessário à instalação e configuração do \textit{iSCSI}.

\begin{figure}[!hbt]
\begin{MyVerbatim}
$ yum install epel-release
\end{MyVerbatim}
\end{figure}

\paragraphnl{Instalação de \textit{Software} Fundamental}

Todos os servidores deverão ter instalados no seu sistema um conjunto de ferramentas e bibliotecas primárias que possibilitem, depois, a instalação de \textit{software} adicional e especializado, como é o caso de um compilador de \textit{C} (\textit{gcc}).

\begin{figure}[!hbt]
\begin{MyVerbatim}
$ yum -y install net-snmp gcc-c++ net-tools bash-completion
\end{MyVerbatim}
\end{figure}
