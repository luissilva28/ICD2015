\subsection{Configuração do Servidor \textit{Storage}}

Para o correto funcionamento da Storage foi essencial a correta configuração dos serviços seguintes:

\begin{itemize}
  \item DRBD
  \item iSCSI Target
\end{itemize}

Optou-se por primeiramente fazer a configuração do SCSI em execução, ao mesmo tempo, nas duas máquinas.

\begin{MyVerbatims}
  \$ yum install scsi-target-utils
  \$ cat /etc/tgt/targets.conf
    <target iqn.2015-02.com.icd:san.targetXX>
      backing-store /dev/sdb
      initiator-address 10.0.1.10
    </target>
  \$ dd if=/dev/zero of=/dev/sdb bs=1024 count=102400
  \$ chkconfig tgtd on
  \$ service tgtd start
\end{MyVerbatims}

O \textit{targetXX} é definido diferentemente em cada uma das máquinas. Neste caso na máquina storage0 é definido o target01 e na storage1 está definido o target02.

Depois de inicializada a configuração do \textit{DRBD} é crucial definir o ficheiro de recurso das máquinas. Depois de terminada a configuração e inicializado o serviço obteve-se uma simulação de RAID1 se verifica, promovendo assim, a alta disponibilidade da infraestrutura.
