\subsection{Configuração do Servidor \textit{Load Balancer}}

Para implementar um sistema de distribuiçao de carga foram utilizados os próximos componentes:

\begin{itemize}
  \item HAProxy
  \item keepalived
\end{itemize}

Para a configuração do \textit{HA-Load balancer} são necessários 2 máquinas físicas para os \textit{Load Balancer}. É necessário definir uma gama de endereços IP das próprias máquinas e saber quais os IP's das máquinas WEB.

O \textit{HAProxy} será utilizado como o \textit{software} de balanceamento de carga entre os servidores e o \textit{keepalived} como uma solução de alta disponibilidade.

Foram então instalados os componentes recorrendo ao \textit{yum} em ambos os nodos.

\begin{MyVerbatims}
  \$ yum install -y haproxy keepalived
\end{MyVerbatims}

Adicionou-se ao haproxy os servidores WEB ativos no momento e alterou-se um parâmetro no \textit{frontend main}.

\begin{MyVerbatims}
  \$vi /etc/haproxy/haproxy.cfg
    frontend  main *:8080
    ....
    backend app
      balance     roundrobin
      server  web0 192.168.73.177:8080 check
      server  web1 192.168.73.211:8080 check
\end{MyVerbatims}

Como não estava a ser possível correr o servidor \textit{Apache} na porta 80 tivemos que utilizar a porta 8080.

Foi ainda definido num ficheiro xml o serviço \textit{HAProxy}.

\begin{MyVerbatims}
  \$ vi /etc/firewalld/services/haproxy.xml
    <?xml version="1.0" encoding="utf-8"?>
    <service>
      <short>HAProxy</short>
      <description>HAProxy load-balancer</description>
      <port protocol="tcp" port="8080"/>
    </service>
\end{MyVerbatims}

Finalmente ativou-se e inicializou-se o serviço nos dois servidores.

\begin{MyVerbatims}
  \$ systemctl enable haproxy
  \$ systemctl start haproxy
\end{MyVerbatims}

De seguida foi necessário configurar o \textit{keepalived} em ambos os nodos anteriores. Alterou-se o ficheiro de configuração e definiu-se quais eram as interfaces a usar, assim como o estado de cada máquina (\textit{Master or Slave}) e o IP Virtual a utilizar para se receber comunicações exteriores.

\begin{MyVerbatims}
  \$ vi /etc/keepalived/keepalived.conf
    vrrp_script chk_haproxy {
      script "killall -0 haproxy" # check the haproxy process
      interval 2 # every 2 seconds
      weight 2
    }

    vrrp_instance VI_1 {
      interface eno33554960 # interface to monitor
      state MASTER # MASTER on haproxy1, BACKUP on haproxy2
      virtual_router_id 51
      priority 101 # 101 on haproxy1, 100 on haproxy2
      virtual_ipaddress {
        192.168.73.49 # virtual ip address
      }
      track_script {
        chk_haproxy
      }
    }
\end{MyVerbatims}

A questão seguinte foi ativar e inicializar o serviço no sistema nas duas máquinas.

\begin{MyVerbatims}
  \$ systemctl enable keepalived
  \$ systemctl start keepalived
\end{MyVerbatims}

Verifica-se então que o Virtual IP foi criado na primeira máquina.

\begin{MyVerbatims}
  ....
  3: eno33554960: <BROADCAST,MULTICAST,UP,LOWER_UP> mtu 1500 qdisc pfifo_fast state UP qlen 1000
    link/ether 00:0c:29:8c:dd:55 brd ff:ff:ff:ff:ff:ff
    inet 192.168.73.200/24 brd 192.168.73.255 scope global eno33554960
       valid_lft forever preferred_lft forever
    inet 192.168.73.49/32 scope global eno33554960
       valid_lft forever preferred_lft forever
  ....
\end{MyVerbatims}
